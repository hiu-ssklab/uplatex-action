\RequirePackage{plautopatch}
\documentclass[a4j,twocolumn,uplatex,dvipdfmx]{jsarticle}
% ----------- 拡張 ----------------------
\usepackage{hyperref}
\usepackage{graphicx}                                          % 画像を使える様にする
\usepackage{syoron}
\usepackage{enumitem}
\setlist[itemize]{topsep=0pt}
\setlist[enumerate]{topsep=0pt,partopsep=0pt,itemsep=0pt,parsep=0pt}
% 行間
\renewcommand{\baselinestretch}{1.1}
% 引用
\usepackage[comma,sort&compress,sectionbib,numbers]{natbib}
\renewcommand{\cite}[1]{\citet{#1}}
\newcommand{\hcite}[1]{\citet{#1}}
% ----------- 拡張ここまで ----------------------
\pagestyle{empty}
%%%%%%%%%%%%%%%%%%%%%%%%%%%%%%%%%%%%%%%%%%%%%%%%%%%
% ここにタイトルを入れる
\title{適当なタイトルを入れてください}
% サブタイトル, 無いなら空欄で良い
\subtitle{}
% 所属ゼミ
\seminar{情報メディア学科 佐々木 洋平ゼミ}
% 学籍番号
\gakuseki{XXXXXXXX}
% ここに著者名を入れる
\author{情報 太郎}
% 日付はどうしようね?
% \date{}
%%%%%%%%%%%%%%%%%%%%%%%%%%%%%%%%%%%%%%%%%%%%%%%%%%%
\begin{document}
\maketitle
\thispagestyle{empty}
%----------- ここから本文 ----------------------
\section{はじめに}

ここには研究業界の流れとか,
本研究の位置付けとかを書く。

最後に本研究の目的を明確にしておくこと。

\section{モデルと実験設定}

何をしたのか, どんな設定なのか, など。
再現性が担保できる様に書くこと。

\section{結果}

結果を書く。
だらだら書かないこと。
表や図があると良い。

\section{議論}

議論とまとめを書く。
「はじめに」で書いた目的に沿った形でまとめること。

\begin{thebibliography}{9}
\bibitem[卒研小論文の様式]{様式2024}
  著者名, ``卒研小論文の様式'', 雑誌名, 2024

\end{thebibliography}

%----------- ここまで本文 ----------------------
\end{document}
